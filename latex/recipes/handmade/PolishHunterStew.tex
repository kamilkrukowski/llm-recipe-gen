\begin{recipe}
[
    preparationtime = {\unit[4]{h}},
    source = Kamil Krukowski,
    portion = \portion{12}
]
{Polish Hunter's Stew I (Chicken Bigos)}

\setRecipeColors{
recipename = orange,
numeration = black
}
    
    \ingredients
    {
            \unit[3]{lbs} & green cabbage \\
            \unit[1]{lb} & sauerkraut \\
            \unit[3]{cups} & chicken broth \\
            \unit[2]{tbsp} & tomato paste\\
            \unit[1]{cup} & dried prunes\\
            \unit[2]{cups} & red wine \\
            \\
            \textbf{Proteins} & \\
            \\
            \unit[\textonehalf]{lb} & kielbasa\\
            \unit[1]{lb} & chicken thigh \\
            \\
            \textbf{Seasoning} & \\
            \\
            2 & bay leaves\\
            \unit[1]{tbsp} & mustard seed\\
            \unit[1]{tsp} & coriander, paprika\\
            \unit[1]{tsp} & caraway, allspice\\
            \unit[1]{tsp} & rosemary, marjoram \\
            \unit[1]{tbsp} & worcestershire sauce \\
    }
    
    \preparation
    { 
        \step Saute seeds and bay leaf until fragrant. Brown kielbasa, set aside.
        \step Steam cabbage leaves until softened.
        \step Combine broth, red wine, tomato paste, prunes, salt, pepper over medium heat. Add cabbage, chicken, and kielbasa. Stew for several hours until the meat is falling apart.
        \step Add sauerkraut, herbs and worcestershire sauce. Let flavours meld briefly. Serve hot or cold with bread.
    }
    
    \hint{
        \begin{itemize}
            \item Recommended Spices: Juniper, Dill, Parsley
            \item Tradition dictates using a variety of proteins, up to 4. The popular addition is pork butt. In general, beef may overwhelm delicate flavours.
            \item For more flavour, consider combining chicken, pork, and bone broths. Mushroom stock is also popular.
        \end{itemize}
    }
    


\end{recipe}
