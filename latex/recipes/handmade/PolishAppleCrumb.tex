\begin{recipe}
[
    preparationtime = {\unit[1.5]{h}},
    portion = \portion{20},
    calory = {\unit[252]{Val}},
    source = Marta Krukowska Feat. Kamil,
]
{Polish Apple Crumb}

\setRecipeColors{
recipename = orange,
numeration = black
}
    \ingredients
    {
        5 & eggs \\
        \unit[3]{cups} & white flour \\
        \unit[1]{cup} & white sugar \\
        \unit[250]{g} & cold unsalted butter \\
        \unit[1\textonehalf]{tbsp} & baking powder \\
        \unit[2]{tsp} & potato starch \\
        \unit[5]{lb} & apple \\
        \\
        \textbf{Apple Spices} & \\
        \\
        \unit[1]{tbsp} & cinnamon powder \\
        \unit[3]{tbsp} & lemon juice \\
        \unit[2]{tsp} & nutmeg
    }
    \preparation
    { 
        \step Separate egg yolks from whites. Incorporate half the sugar with the egg yolks.
        \step Grate the cold butter into a bowl, working quickly to avoid melt.
        \step Knead baking powder, flour, grated butter, and the sugar egg yolk mixture into a dough. Split dough into two portions, refrigerate one portion.
        \step Flatten the second portion to cover a casserole pan for the cake base. Prick the flattened dough with the tines of a fork until the entire surface is dotted. Bake at 350f for 15 minutes or until golden brown, leave to cool.
        \step Peel apples, cut into thin wedges. Slice wedges in half. In a covered pan, braise apple slices with lemon juice, cinnamon, nutmeg, and a touch of water over medium heat until softened. Strain and set aside.
        \step Using a mixer, create a meringue. Beat egg whites into stiff peaks while slowly folding in sugar and potato starch.
        \step Layer the braised apples over the baked cracker base in the casserole dish. Apply meringue evenly across the apple layer to bind the crumbs in the next step. Grate the previously refrigerated dough over the top to form a crispy topping.
        \step Bake for 30 minutes at 350f. Serve hot or cold, optionally garnish with vanilla ice cream.
    }
    \hint{Consider cooking the apples to two degrees of doneness: Half like apple sauce and half barely softened. Apples should be a mix of sour/tart baking varieties, ie Granny Smith and one more.}

\end{recipe}